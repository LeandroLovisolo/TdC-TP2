
\documentclass[a4paper, 10pt, twoside]{article}

\usepackage[top=1in, bottom=1in, left=1in, right=1in]{geometry}
\usepackage[utf8]{inputenc}
\usepackage[spanish, es-ucroman, es-noquoting]{babel}
\usepackage{setspace}
\usepackage{fancyhdr}
\usepackage{lastpage}
\usepackage{amsmath}
\usepackage{amsfonts}
\usepackage{amsthm}
\usepackage{verbatim}
\usepackage{graphicx}
\usepackage{float}
\usepackage{enumitem} % Provee macro \setlist
\usepackage{tabularx}
\usepackage{multirow}
\usepackage{hyperref}
\usepackage{bytefield}
\usepackage[toc, page]{appendix}


%%%%%%%%%% Configuración de Fancyhdr - Inicio %%%%%%%%%%
\pagestyle{fancy}
\thispagestyle{fancy}
\lhead{Trabajo Práctico 2 · Teoría de las Comunicaciones}
\rhead{Delgado · Lovisolo · Petaccio}
\renewcommand{\footrulewidth}{0.4pt}
\cfoot{\thepage /\pageref{LastPage}}

\fancypagestyle{caratula} {
   \fancyhf{}
   \cfoot{\thepage /\pageref{LastPage}}
   \renewcommand{\headrulewidth}{0pt}
   \renewcommand{\footrulewidth}{0pt}
}
%%%%%%%%%% Configuración de Fancyhdr - Fin %%%%%%%%%%


%%%%%%%%%% Miscelánea - Inicio %%%%%%%%%%
% Evita que el documento se estire verticalmente para ocupar el espacio vacío
% en cada página.
\raggedbottom

% Separación entre párrafos.
\setlength{\parskip}{0.5em}

% Separación entre elementos de listas.
\setlist{itemsep=0.5em}

% Asigna la traducción de la palabra 'Appendices'.
\renewcommand{\appendixtocname}{Apéndices}
\renewcommand{\appendixpagename}{Apéndices}
%%%%%%%%%% Miscelánea - Fin %%%%%%%%%%


%%%%%%%%%% Insertar estadísticas - Inicio %%%%%%%%%%
\newcommand{\estadisticas}[3]{
  \begin{figure}[H]
    \small
    \verbatiminput{#1}
    \normalsize
    \caption{#2}
    \label{#3}
  \end{figure}
}
%%%%%%%%%% Insertar estadísticas - Fin %%%%%%%%%%


%%%%%%%%%% Insertar gráfico - Inicio %%%%%%%%%%
\newcommand{\grafico}[3]{
  \begin{figure}[H]
    \includegraphics[type=pdf,ext=.pdf,read=.pdf]{#1}
    \caption{#2}
    \label{#3}
  \end{figure}
}
%%%%%%%%%% Insertar gráfico - Fin %%%%%%%%%%


%%%%%%%%%% Nombres de las universidades - Inicio %%%%%%%%%%
\newcommand{\oxford}{University of Oxford}
\newcommand{\sydney}{The University of Sydney}
\newcommand{\must}{Malaysia University of Science and Technology}
%%%%%%%%%% Nombres de las universidades - Fin %%%%%%%%%%


\begin{document}


%%%%%%%%%%%%%%%%%%%%%%%%%%%%%%%%%%%%%%%%%%%%%%%%%%%%%%%%%%%%%%%%%%%%%%%%%%%%%%%
%% Carátula                                                                  %%
%%%%%%%%%%%%%%%%%%%%%%%%%%%%%%%%%%%%%%%%%%%%%%%%%%%%%%%%%%%%%%%%%%%%%%%%%%%%%%%


\thispagestyle{caratula}

\begin{center}

\includegraphics[height=2cm]{DC.png} 
\hfill
\includegraphics[height=2cm]{UBA.jpg} 

\vspace{2cm}

Departamento de Computación,\\
Facultad de Ciencias Exactas y Naturales,\\
Universidad de Buenos Aires

\vspace{4cm}

\begin{Huge}
Trabajo Práctico 2
\end{Huge}

\vspace{0.5cm}

\begin{Large}
Teoría de las Comunicaciones
\end{Large}

\vspace{1cm}

Primer Cuatrimestre de 2014

\vspace{4cm}

\begin{tabular}{|c|c|c|}
\hline
Apellido y Nombre & LU & E-mail\\
\hline
Delgado, Alejandro N.  & 601/11 & nahueldelgado@gmail.com\\
Lovisolo, Leandro      & 645/11 & leandro@leandro.me\\
Petaccio, Lautaro José & 443/11 & lausuper@gmail.com\\
\hline
\end{tabular}

\end{center}

\newpage


%%%%%%%%%%%%%%%%%%%%%%%%%%%%%%%%%%%%%%%%%%%%%%%%%%%%%%%%%%%%%%%%%%%%%%%%%%%%%%%
%% Índice                                                                    %%
%%%%%%%%%%%%%%%%%%%%%%%%%%%%%%%%%%%%%%%%%%%%%%%%%%%%%%%%%%%%%%%%%%%%%%%%%%%%%%%


\tableofcontents

\newpage


%%%%%%%%%%%%%%%%%%%%%%%%%%%%%%%%%%%%%%%%%%%%%%%%%%%%%%%%%%%%%%%%%%%%%%%%%%%%%%%
%% Introducción                                                              %%
%%%%%%%%%%%%%%%%%%%%%%%%%%%%%%%%%%%%%%%%%%%%%%%%%%%%%%%%%%%%%%%%%%%%%%%%%%%%%%%


\section{Introducción}

Presentamos una heurística y su análisis de efectividad para realizar la detección de enlaces submarinos en la traza de rutas entre dos hosts conectados a internet, utilizando los datos estadísticos resultantes del round trip time (RTT) conseguidos mediante una implementación propia de la herramienta traceroute, comunmente encontrada en los SO.

Para el análisis estadístico, los RTT son estudiados como el \textit{z-score} o valor standard (ZRTT) respecto a las variaciones de los valores RTT promedios entre dos nodos contínuos. La utilización del ZRTT relativo como herramienta estadística presenta una manera detallada de identificar variaciones de tiempo entre enlaces, pudiendo identificar nodos con una diferencia en promedio de RTT mayor al promedio, como tendría un enlace submarino debido a la distancia que recorren los datos.


%%%%%%%%%%%%%%%%%%%%%%%%%%%%%%%%%%%%%%%%%%%%%%%%%%%%%%%%%%%%%%%%%%%%%%%%%%%%%%%
%% Desarrollo                                                                %%
%%%%%%%%%%%%%%%%%%%%%%%%%%%%%%%%%%%%%%%%%%%%%%%%%%%%%%%%%%%%%%%%%%%%%%%%%%%%%%%


\section{Desarrollo}

Se implementó una herramienta en lenguaje Python para medir los \emph{round-trip times} (RTT) hacia el host destino y cada hop intermedio durante una cantidad de tiempo determinada por el usuario. La herramienta se basa en el protocolo ICMP \cite{rfc-792} tanto para descubrir la cantidad de hops y los gateways en cada hop hacia el host destino, como para medir los RTT hacia el host destino y cada hop intermedio (de manera similar a las herramientas \texttt{traceroute} \cite{wiki-traceroute} y \texttt{ping} \cite{wiki-ping}, respectivamente.)

La herramienta hace uso la biblioteca Scapy \cite{scapy} para la creación y comunicación de paquetes ICMP.


\subsection{Medición de RTT hacia el host destino o un hop intermedio}

Una medición consiste en enviar un paquete ICMP de tipo Echo Request al host destino, asignándole al paquete algún \emph{time-to-live} (TTL) entre 1 y 30 inclusive, y tomar el tiempo que transcurre desde que se envía el paquete hasta que se recibe una respuesta. Las respuestas usualmente son de alguno de los siguientes tipos:

\begin{itemize}
  \item Un paquete ICMP de tipo Echo Reply en caso que el paquete emitido alcanzara el host destino, ó
  \item Un paquete ICMP de tipo Time Exceeded en caso que el paquete agotara su TTL antes de llegar al host destino.
\end{itemize}

Es posible recibir respuestas de otros tipos, como por ejemplo paquetes ICMP de tipo Destination Unreachable en el caso que no se haya podido despachar el paquete a su destino por algún motivo, pero la herramienta ignora cualquier respuesta que no sea de los dos tipos anteriores.

Para medir el RTT hacia el host destino, basta con enviarle un paquete a dicho host con un TTL lo suficientemente grande para asegurar que su TTL no se agote durante el envío del paquete, esperar hasta recibir un paquete ICMP de tipo Echo Reply proveniente del host destino y registrar el tiempo transcurrido.

Para medir el RTT hacia el $i$-ésimo hop en la ruta al host destino, se le asigna al paquete un TTL de valor $i$. Esto produce que el paquete agote su TTL al llegar al $i$-éstimo host, a lo cual éste responde con un paquete ICMP de tipo Time Exceeded. Finalmente se registra el tiempo transcurrido.

Tanto cuando se mide el RTT hacia el host destino o hacia un hop intermedio puede ocurrir que no se reciba ninguna respuesta, por ejemplo cuando el host o algún gateway está detrás de un firewall que bloquea el protocolo ICMP. Para evitar quedar esperando una respuesta durante una cantidad de tiempo indefinida, la herramienta descarta la medición si al cabo de un segundo no se recibio una respuesta.


\subsection{Realizando múltiples mediciones en paralelo}

Con el objetivo de maximizar el número total de mediciones realizadas y distribuir el impacto de picos de retraso en la conexión a internet entre las mediciones de RTT hacia todos los hops, la herramienta realiza mediciones hacia todos los hops de forma simultánea.

Las mediciones se hacen por baches: en un determinado momento se envían 30 paquetes al host destino, uno por cada TTL entre 1 y 30 y todos con TTL distinto, y se espera o bien hasta recibir las respuestas de todos los paquetes enviados, o bien hasta que transcurra un segundo y se den por perdidas las mediciones para las que no se recibieron respuestas. A continuación se registra el RTT hacia cada hop computando la diferencia entre el tiempo de recepción de una respuesta y el tiempo de envío del paquete de tipo Echo Request que la originó. Luego de esto se procede al siguiente bache de mediciones, o se finaliza en caso de haber excedido el límite de tiempo de medición determinado por el usuario.

Para poder distinguir qué paquete produjo cada respuesta recibida, la herramienta le asigna un identificador único a cada paquete ICMP de tipo Echo Request emitido usando el campo \emph{Identifier} (figura \ref{fig:icmp-echo-request}) de manera de luego poder obtener el valor del campo TTL de un paquete ICMP de tipo Echo Request conociendo el valor de su campo \emph{Identifier}.

\begin{figure}[H]
  \vspace{2em}
  \begin{center}
    \begin{bytefield}[bitwidth=1.1em]{32}
      \bitheader{0-31} \\
      \bitbox{8}{Type = 8} & \bitbox{8}{Code = 0} & \bitbox{16}{Header Checksum} \\
      \bitbox{16}{Identifier} & \bitbox{16}{Sequence Number} \\
      \bitbox{32}{\emph{Datos}}
    \end{bytefield}
  \end{center}
  \caption{Paquete ICMP de tipo Echo Request}
  \label{fig:icmp-echo-request}
\end{figure}

En el caso que un paquete ICMP de tipo Echo Request haya llegado al host destino, éste contesta enviando un paquete ICMP de tipo Echo Reply (figura \ref{fig:icmp-echo-reply}.) Este paquete también tiene un campo \emph{Identifier}, que conserva el valor del mismo campo en el paquete ICMP de tipo Echo Request que lo originó.

\begin{figure}[H]
  \vspace{2em}
  \begin{center}
    \begin{bytefield}[bitwidth=1.1em]{32}
      \bitheader{0-31} \\
      \bitbox{8}{Type = 0} & \bitbox{8}{Code = 0} & \bitbox{16}{Header Checksum} \\
      \bitbox{16}{Identifier} & \bitbox{16}{Sequence Number} \\
      \bitbox{32}{\emph{Datos}}
    \end{bytefield}
  \end{center}
  \caption{Paquete ICMP de tipo Echo Reply}
  \label{fig:icmp-echo-reply}
\end{figure}

Cuando un paquete (no necesariamente ICMP) agota su TTL antes de llegar al host destino, el último gateway al que llegó dicho paquete envía al host origen un paquete ICMP de tipo Time Exceeded (figura \ref{fig:icmp-time-exceeded}.) Éste paquete incluye el header IP y los primeros 8 bytes de datos del datagrama que agotó su TTL.

\begin{figure}[H]
  \vspace{2em}
  \begin{center}
    \begin{bytefield}[bitwidth=1.1em]{32}
      \bitheader{0-31} \\
      \bitbox{8}{Type = 11} & \bitbox{8}{Code} & \bitbox{16}{Header Checksum} \\
      \bitbox{32}{\emph{No utilizado}} \\
      \wordbox{2}{\emph{Header IP y los primeros 8 bytes de datos del datagrama original} \\ $\vdots$}
    \end{bytefield}
  \end{center}
  \caption{Paquete ICMP de tipo Time Exceeded}
  \label{fig:icmp-time-exceeded}
\end{figure}

En particular, cuando el paquete que agotó su TTL es un paquete ICMP de tipo Echo Request, su header ICMP completo se incluye como parte de los 8 bytes de datos del datagrama original, del cual se puede extraer el valor del campo \emph{Identifier} (ver figura \ref{fig:icmp-time-exceeded-echo-request}.) 

\begin{figure}[H]
  \vspace{2em}
  \begin{center}
    \begin{bytefield}[bitwidth=1.1em]{32}
      \bitheader{0-31} \\
      \bitbox{8}{Type = 11} & \bitbox{8}{Code} & \bitbox{16}{Header Checksum} \\
      \bitbox{32}{\emph{No utilizado}} \\
      \bitbox{32}{\emph{Header IP del paquete original}} \\
      \begin{rightwordgroup}{Header ICMP \\ del paquete \\ original}
        \bitbox{8}{Type = 8} & \bitbox{8}{Code = 0} & \bitbox{16}{Header Checksum} \\
        \bitbox{16}{Identifier} & \bitbox{16}{Sequence Number}
      \end{rightwordgroup}
    \end{bytefield}
  \end{center}
  \caption{Paquete ICMP de tipo Time Exceeded como respuesta a otro paquete ICMP de tipo Echo Request}
  \label{fig:icmp-time-exceeded-echo-request}
\end{figure}

La herramienta entonces recibe paquetes ICMP de tipo Echo Reply o Time Exceeded, y para cada paquete, extrae el valor del campo \emph{Identifier}, que coincide con el valor del mismo campo en el paquete ICMP de tipo Echo Request que lo originó. A partir del valor del campo \emph{Identifier}, se obtiene el TTL del paquete que originó la respuesta y se almacena el RTT medido junto al resto de las mediciones del hop cuyo número coincide con dicho TTL.


\subsection{Registro de mediciones y estadísticas computadas}

Al finalizar la ejecución, la herramienta opcionalmente guarda a disco todas las respuestas recibidas que no fueron descartadas, junto al RTT medido para cada respuesta. En concreto, para respuesta recibida, se guardan los siguientes datos: TTL del paquete que la originó, IP del host que emitió la respuesta, tipo de la respuesta (valor del campo \emph{Type} del header ICMP) y RTT expresado en milisegundos.

Junto a la herramienta desarrollada se provee una utilidad para leer los datos guardados a discos y generar estadísticas. Las estadísticas generadas son, para cada hop, RTT promedio (el RTT promedio de todos los paquetes recibidos provenientes de ese hop) y ZRTT.

La salida de dicha herramienta se incluye en la sección \ref{sec:resultados}.


\subsection{Experimentos realizados}

Se eligieron como hosts destino los servidores web de tres universidades ubicadas en continentes distintos entre sí y respecto del continente desde el que se realizaron las mediciones, con la esperanza de atravezar uno o más enlaces submarinos distintos en cada traza obtenida. Las mediciones se realizaron desde Buenos Aires, Argentina, y las universidades elegidas fueron \oxford, \sydney\ y \must\ (Europa, Oceanía y Asia, respectivamente.)

Una vez determinados los hosts destino, se hicieron experimentos con versiones anteriores (y más simples) de la herramienta desarrollada. Tras cada experimento se modificó la herramienta para realizar mediciones más precisas, reflejando las conclusiones obtenidas en los experimentos anteriores. Este proceso se repitió hasta converger en la herramienta presentada en este trabajo.


\subsubsection{Experimento 1: múltiples mediciones en paralelo, un único hop por vez}

En esta primera iteración de la herramienta se enviaban simultáneamente 100 paquetes ICMP de tipo Echo Request al host destino con TTL 1 y se esperaba hasta recibir las 100 respuestas correspondientes, o bien hasta que transcurriera un segundo desde que se enviaron los paquetes. En este último caso, se descartan las mediciones correspondientes a los paquetes para los que no se recibieron respuestas. Luego se repite este proceso para los valores de TTL de 2 a 30 inclusive.

Esta técnica resultó estar muy sujeta a la congestión de la red en el momento que se tomaron las mediciones. Por ejemplo, si en el instante que se envían los paquetes para el $i$-ésimo TTL la conexión a internet de la computadora desde la que se realiza la medición sufre una congestión, pero la conexión se normaliza para el instante en el que se envían los paquetes para el $(i+1)$-ésimo TTL, el RTT promedio para el $i$-ésimo hop puede resultar muy superior al del $(i+1)$-ésimo hop. Esta anomalía puede producir un falso positivo en la etapa de detección de enlaces submarinos más adelante en el análisis.

Luego de repetir varias veces las mediciones para cada host destino y observar resultados muy distintos entre medición y medición para un mismo host, se decidió modificar la herramienta de manera de esparcir uniformemente en el tiempo las mediciones para cada TTL con la esperanza de suavizar las anomalías producidas por variaciones en la carga de la conexión a internet.


%%%%%%%%%%%%%%%%%%%%%%%%%%%%%%%%%%%%%%%%%%%%%%%%%%%%%%%%%%%%%%%%%%%%%%%%%%%%%%%
%% Resultados                                                                %%
%%%%%%%%%%%%%%%%%%%%%%%%%%%%%%%%%%%%%%%%%%%%%%%%%%%%%%%%%%%%%%%%%%%%%%%%%%%%%%%


\section{Resultados}
\label{sec:resultados}


\subsection{\oxford}

\estadisticas{statistics-www.ox.ac.uk.txt}
             {Traza hacia \oxford}
             {fig:trace-oxford}

\grafico{map-www.ox.ac.uk}
        {Ruta hacia \oxford}
        {fig:map-oxford}

\grafico{rtt-www.ox.ac.uk}
        {RTT de los gateways de la ruta hacia \oxford}
        {fig:rtt-oxford}

\grafico{zrtt-www.ox.ac.uk}
        {ZRTT de los gateways de la ruta hacia \oxford}
        {fig:zrtt-oxford}


\subsection{\sydney}

\estadisticas{statistics-www.sydney.edu.au.txt}
             {Traza hacia \sydney}
             {fig:trace-sydney}

\grafico{map-www.sydney.edu.au}
        {Ruta hacia \sydney}
        {fig:map-sydney}

\grafico{rtt-www.sydney.edu.au}
        {RTT de los gateways de la ruta hacia \sydney}
        {fig:rtt-sydney}

\grafico{zrtt-www.sydney.edu.au}
        {ZRTT de los gateways de la ruta hacia \sydney}
        {fig:zrtt-sydney}


\subsection{\must}

\estadisticas{statistics-www.must.edu.my.txt}
             {Traza hacia \must}
             {fig:trace-must}

\grafico{map-www.must.edu.my}
        {Ruta hacia \must}
        {fig:map-must}

\grafico{rtt-www.must.edu.my}
        {RTT de los gateways de la ruta hacia \must}
        {fig:rtt-must}

\grafico{zrtt-www.must.edu.my}
        {ZRTT de los gateways de la ruta hacia \must}
        {fig:zrtt-must}


%%%%%%%%%%%%%%%%%%%%%%%%%%%%%%%%%%%%%%%%%%%%%%%%%%%%%%%%%%%%%%%%%%%%%%%%%%%%%%%
%% Discusión                      			                                     %%
%%%%%%%%%%%%%%%%%%%%%%%%%%%%%%%%%%%%%%%%%%%%%%%%%%%%%%%%%%%%%%%%%%%%%%%%%%%%%%%


\section{Discusión}

Teniendo como referencia los resultados obtenidos, notamos que ocurrieron algunos fenónemos en común que creemos vale la pena mencionar.

\subsection{Primer nodo externo}
Podemos notar, en los resultados de todas las universidades analizadas, que el segundo nodo, el primer nodo que no es nuestro gateway, posee un RTT promedio alto en comparación a los nodos más próximos a éste. Podemos deducir que esto se debe a que este router tiene una prioridad baja para las respuestas a paquetes ICMP, haciendo que las respuestas tarden más de lo esperado.

\subsection{Posible error de localización}
Otro punto a tener en cuenta, también visible en los resultados de todas las universidades, es que la biblioteca elegida para la geolocalización sitúa muy probablemente la localización de los IP según la procedencia de la compañía que realice el enlace entre países. Pueden notarse en los resultados hops con bajo ZRTT relativo entre distintos países y luego el próximo nodo dentro del segundo país tiene un ZRTT relativo muy alto, indicando probablemente que se trata de un enlace de gran distancia.

\subsection{Promedios de RTT}
Un dato estadístico anómalo general es el de obtener, para un determinado nodo, un RTT absoluto menor al RTT absoluto del nodo anterior, lo que significaría que llegar a este nodo toma menos tiempo que llegar al anterior. Esta anomalía se debe a que los valores del RTT absolutos se calculan mediante el promedio de los RTT obtenidos para cada nodo, pudiendo el paquete ICMP haber tomado caminos diferentes y habiendo conseguido llegar de manera apenas más rápida en promedio.

\subsection{Posibles enlaces submarinos}
En la traza a la universidad de Oxford podemos observar en el gráfico del ZRTT relativo como la IP 67.16.134.218 asignado a Estados Unidos obtiene un ZRTT relativo alto estando rodeado por dos nodos cuyos ZRTT son bajos y su nodo anterior sufre del problema general de la geolocalización del cuál hablamos anteriormente (el nodo debería pertenecer a un router en Argentina), podemos decir que el IP mencionado pertenece al primer router luego de un enlace submarino.
 
Podemos observar también en el gráfico del ZRTT relativo de esta universidad como existen varios enlaces (4.69.138.123 y 4.69.202.65) de los cuales no es posible deducir con certeza la causa de sus altos valores, pero podemos especular de que, algún router posee una prioridad baja para contestar paquetes ICMP o que las IP sufren del problema de geolocalización indicado y estos saltos son entre Estados Unidos y algún país europeo y luego de el continente europeo a Reino Unido o posiblemente, una combinación de ambos (un salto a Reino unido y un router con prioridad de contestación baja). Esta deducción surge de que el análisis de los ZRTT relativos de los nodos siguientes muestra valores para los ZRTT muy bajos, incluso cuando la geolocalización muestra el cambio de paises.
 
La ruta a \sydney \ muestra un gráfico de ZRRT relativos satisfactorio en cuánto al análisis de saltos submarinos. Podemos notar la IP del router 67.16.139.18 que sufre del problema de geolocalización y que al obtener un ZRTT relativo alto en relación a sus nodos vecinos y al estar próximo de Argentina, es posible identificarlo como salto submarino. El próximo salto notable es el de Estado Unidos a Australia de IP 202.158.194.172, donde podemos ver que esta IP no sufre del problema de geolocalización y marca un ZRTT relativo alto entre nodos cercanos además del cambio de país. Los demás routers del recorrido, a excepción del caso general del segundo nodo en la conexión, muestran ZRTT relativos esperables y bajos al no ser saltos submarinos, dejando como distinguidos los IP mencionados.
 
Por último, la traza obtenida a \must \ muestra en su gráfico de ZRRT relativos, además del caso general del segundo nodo con alto ZRTT, 3 IP, 67.17.192.6, 203.208.183.145 y 203.208.153.166 los cuales podemos tomar como saltos submarinos. La IP 67.17.192.6 cae en el caso de geolocalización errónea y correspondería al salto de Argentina a Estados Unidos y las IP 203.208.183.145 y 203.208.153.166 que también sufren de lo mismo y que estimamos que sus ZRTT relativos representan un salto de Estados Unidos a Singapore y de Singapore a Malasia respectivamente. Los demás IP tienen ZRTT relativos bajos, indicando comunicaciones entre nodos cercanos.

\subsection{Heurística para detección de enlaces submarino}
Basándonos en el análisis realizado sobre la experimentación, proponemos como umbral en las mediciones de los ZRTT relativos para la detección de enlaces sumbarinos el valor 1. El umbral propuesto creemos que es suficiente para detectar grandes variaciones en relación al desvío estándar de RTT entre nodos.

Como lo planteamos anteriormente, los enlaces submarinos y los routers que asignan prioridad baja a las respuestas de paquetes ICMP muestran ambos un ZRTT alto, pero con la diferencia de que los routers que asignan una prioridad diferente hacen que el nodo siguiente tenga un ZRTT más bajo que el resto. Si bien se destacan del resto, usar únicamente un umbral positivo sobre los ZRTT presenta problemas a la hora de decidir si realmente pertenecen a un enlace submarino.

%%%%%%%%%%%%%%%%%%%%%%%%%%%%%%%%%%%%%%%%%%%%%%%%%%%%%%%%%%%%%%%%%%%%%%%%%%%%%%%
%% Conclusión                                                                %%
%%%%%%%%%%%%%%%%%%%%%%%%%%%%%%%%%%%%%%%%%%%%%%%%%%%%%%%%%%%%%%%%%%%%%%%%%%%%%%%


\section{Conclusión}

Podemos concluir en que la heurística basada en ZRTT's relativos funciona en casos donde es posible diferenciar un salto submarino de un router con prioridad baja de contestación de paquetes ICMP. En los casos donde el camino posee una variedad de routers que hacen diferencia en el tiempo de contestación, la identificación de los saltos puede volverse dificultosa y es probable que requiera de una base de geolocalización precisa para la resolución del problema.

Debido al problema de clasificación de nodos (nodo con prioridad baja para paquetes ICMP o enlace submarino) planteamos como mejora a futuro para la heurística, la utilización de un umbral negativo para las mediciones de los ZRTT relativos, el cual parece efectivo para descartar los casos en los que se obtengan routers con prioridad baja, en especial el del primer nodo externo.


%%%%%%%%%%%%%%%%%%%%%%%%%%%%%%%%%%%%%%%%%%%%%%%%%%%%%%%%%%%%%%%%%%%%%%%%%%%%%%%
%% Referencias                                                               %%
%%%%%%%%%%%%%%%%%%%%%%%%%%%%%%%%%%%%%%%%%%%%%%%%%%%%%%%%%%%%%%%%%%%%%%%%%%%%%%%


\begin{thebibliography}{9}

\bibitem{scapy}
	\emph{Scapy Project}.
	\url{http://www.secdev.org/projects/scapy}, 
	Mayo de 2014.

\bibitem{rfc-792}
  \emph{RFC 792: Internet Control Message Protocol}.
  \url{http://tools.ietf.org/html/rfc792}.

\bibitem{wiki-traceroute}
  \emph{Traceroute}.
  \url{http://en.wikipedia.org/wiki/Traceroute},
  Mayo de 2014.

\bibitem{wiki-ping}
  \emph{Ping (network utility) (Artículo en Wikipedia)}.
  \url{http://en.wikipedia.org/wiki/Ping_(networking_utility)},
  Mayo de 2014.

\bibitem{wiki-icmp}
  \emph{Internet Control Message Protocol (Artículo en Wikipedia)}.
  \url{http://en.wikipedia.org/wiki/Internet_Control_Message_Protocol},
  Mayo de 2014.
	
\end{thebibliography}


\end{document}