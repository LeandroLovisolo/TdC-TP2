
\documentclass[a4paper, 10pt, twoside]{article}

\usepackage[top=1in, bottom=1in, left=1in, right=1in]{geometry}
\usepackage[utf8]{inputenc}
\usepackage[spanish, es-ucroman, es-noquoting]{babel}
\usepackage{setspace}
\usepackage{fancyhdr}
\usepackage{lastpage}
\usepackage{amsmath}
\usepackage{amsfonts}
\usepackage{amsthm}
\usepackage{verbatim}
\usepackage{graphicx}
\usepackage{float}
\usepackage{enumitem} % Provee macro \setlist
\usepackage{tabularx}
\usepackage{multirow}
\usepackage{hyperref}
\usepackage{bytefield}
\usepackage[toc, page]{appendix}


%%%%%%%%%% Configuración de Fancyhdr - Inicio %%%%%%%%%%
\pagestyle{fancy}
\thispagestyle{fancy}
\lhead{Trabajo Práctico 2 · Teoría de las Comunicaciones}
\rhead{Delgado · Lovisolo · Petaccio}
\renewcommand{\footrulewidth}{0.4pt}
\cfoot{\thepage /\pageref{LastPage}}

\fancypagestyle{caratula} {
   \fancyhf{}
   \cfoot{\thepage /\pageref{LastPage}}
   \renewcommand{\headrulewidth}{0pt}
   \renewcommand{\footrulewidth}{0pt}
}
%%%%%%%%%% Configuración de Fancyhdr - Fin %%%%%%%%%%


%%%%%%%%%% Miscelánea - Inicio %%%%%%%%%%
% Evita que el documento se estire verticalmente para ocupar el espacio vacío
% en cada página.
\raggedbottom

% Separación entre párrafos.
\setlength{\parskip}{0.5em}

% Separación entre elementos de listas.
\setlist{itemsep=0.5em}

% Asigna la traducción de la palabra 'Appendices'.
\renewcommand{\appendixtocname}{Apéndices}
\renewcommand{\appendixpagename}{Apéndices}
%%%%%%%%%% Miscelánea - Fin %%%%%%%%%%


%%%%%%%%%% Insertar estadísticas - Inicio %%%%%%%%%%
\newcommand{\estadisticas}[3]{
  \begin{figure}[H]
    \verbatiminput{#1}
    \caption{#2}
    \label{#3}
  \end{figure}
}
%%%%%%%%%% Insertar estadísticas - Fin %%%%%%%%%%


%%%%%%%%%% Insertar gráfico - Inicio %%%%%%%%%%
\newcommand{\grafico}[3]{
  \begin{figure}[H]
    \includegraphics[type=pdf,ext=.pdf,read=.pdf]{#1}
    \caption{#2}
    \label{#3}
  \end{figure}
}
%%%%%%%%%% Insertar gráfico - Fin %%%%%%%%%%


%%%%%%%%%% Nombres de las universidades - Inicio %%%%%%%%%%
\newcommand{\oxford}{University of Oxford}
\newcommand{\sydney}{The University of Sydney}
\newcommand{\must}{Malaysia University of Science and Technology}
%%%%%%%%%% Nombres de las universidades - Fin %%%%%%%%%%


\begin{document}


%%%%%%%%%%%%%%%%%%%%%%%%%%%%%%%%%%%%%%%%%%%%%%%%%%%%%%%%%%%%%%%%%%%%%%%%%%%%%%%
%% Carátula                                                                  %%
%%%%%%%%%%%%%%%%%%%%%%%%%%%%%%%%%%%%%%%%%%%%%%%%%%%%%%%%%%%%%%%%%%%%%%%%%%%%%%%


\thispagestyle{caratula}

\begin{center}

\includegraphics[height=2cm]{DC.png} 
\hfill
\includegraphics[height=2cm]{UBA.jpg} 

\vspace{2cm}

Departamento de Computación,\\
Facultad de Ciencias Exactas y Naturales,\\
Universidad de Buenos Aires

\vspace{4cm}

\begin{Huge}
Trabajo Práctico 2
\end{Huge}

\vspace{0.5cm}

\begin{Large}
Teoría de las Comunicaciones
\end{Large}

\vspace{1cm}

Primer Cuatrimestre de 2014

\vspace{4cm}

\begin{tabular}{|c|c|c|}
\hline
Apellido y Nombre & LU & E-mail\\
\hline
Delgado, Alejandro N.  & 601/11 & nahueldelgado@gmail.com\\
Lovisolo, Leandro      & 645/11 & leandro@leandro.me\\
Petaccio, Lautaro José & 443/11 & lausuper@gmail.com\\
\hline
\end{tabular}

\end{center}

\newpage


%%%%%%%%%%%%%%%%%%%%%%%%%%%%%%%%%%%%%%%%%%%%%%%%%%%%%%%%%%%%%%%%%%%%%%%%%%%%%%%
%% Índice                                                                    %%
%%%%%%%%%%%%%%%%%%%%%%%%%%%%%%%%%%%%%%%%%%%%%%%%%%%%%%%%%%%%%%%%%%%%%%%%%%%%%%%


\tableofcontents

\newpage


%%%%%%%%%%%%%%%%%%%%%%%%%%%%%%%%%%%%%%%%%%%%%%%%%%%%%%%%%%%%%%%%%%%%%%%%%%%%%%%
%% Introducción                                                              %%
%%%%%%%%%%%%%%%%%%%%%%%%%%%%%%%%%%%%%%%%%%%%%%%%%%%%%%%%%%%%%%%%%%%%%%%%%%%%%%%


\section{Introducción}

En este trabajo estudiamos un método para detectar enlaces submarinos en la traza de paquetes entre dos hosts conectados a internet.


%%%%%%%%%%%%%%%%%%%%%%%%%%%%%%%%%%%%%%%%%%%%%%%%%%%%%%%%%%%%%%%%%%%%%%%%%%%%%%%
%% Desarrollo                                                                %%
%%%%%%%%%%%%%%%%%%%%%%%%%%%%%%%%%%%%%%%%%%%%%%%%%%%%%%%%%%%%%%%%%%%%%%%%%%%%%%%


\section{Desarrollo}

La herramienta implementada mide los RTT hacia el host destino y cada hop intermedio durante una cantidad de tiempo dada.

Cada medición se realiza enviando paquetes ICMP de tipo Echo Request al host destino, asignándole al paquete algún TTL entre 1 y 30 inclusive. Si al cabo de un segundo no se recibe ninguna respuesta, se da por perdida esa medición.

Las mediciones se hacen por baches: en un determinado momento se envían 30 paquetes al host destino, uno por cada TTL en el rango mencionado y todos con TTL distinto, y se espera o bien hasta recibir las respuestas de todos los paquetes enviados, o bien hasta que transcurra un segundo; lo que ocurra primero. A continuación se registra el RTT hacia cada hop computando la diferencia entre el tiempo de recepción de una respuesta y el tiempo de envío del paquete de tipo Echo Request que la originó. Luego de esto se procede al siguiente bache de mediciones, o se finaliza en caso de exceder el tiempo de medición determinado.

Para poder identificar qué paquete produjo cada respuesta recibida se hace uso del campo \emph{Identifier} de los paquetes ICMP de tipo Echo Request (figura \ref{fig:icmp-echo-request}.)

\begin{figure}[h]
  \vspace{1em}
  \begin{center}
    \begin{bytefield}[bitwidth=1.1em]{32}
      \bitheader{0-31} \\
      \bitbox{8}{Type = 8} & \bitbox{8}{Code = 0} & \bitbox{16}{Header Checksum} \\
      \bitbox{16}{Identifier} & \bitbox{16}{Sequence Number} \\
      \bitbox{32}{\emph{Datos}}
    \end{bytefield}
    \caption{Paquete ICMP de tipo Echo Request}
  \end{center}
  \label{fig:icmp-echo-request}
\end{figure}

En el caso que el paquete haya llegado al host destino, éste contesta enviando un paquete ICMP de tipo Echo Reply (figura \ref{fig:icmp-echo-reply}.) Este paquete también tiene un campo \emph{Identifier}, que conserva el valor del mismo campo en el paquete ICMP de tipo Echo Request que lo originó.

\begin{figure}[h]
  \vspace{1em}
  \begin{center}
    \begin{bytefield}[bitwidth=1.1em]{32}
      \bitheader{0-31} \\
      \bitbox{8}{Type = 0} & \bitbox{8}{Code = 0} & \bitbox{16}{Header Checksum} \\
      \bitbox{16}{Identifier} & \bitbox{16}{Sequence Number} \\
      \bitbox{32}{\emph{Datos}}
    \end{bytefield}
    \caption{Paquete ICMP de tipo Echo Reply}
  \end{center}
  \label{fig:icmp-echo-reply}
\end{figure}

Cuando un paquete (no necesariamente ICMP) rumbo al host destino agota su \emph{time to live}, el último gateway al que llegó el paquete contesta con un paquete ICMP de tipo Time Exceeded (figura \ref{fig:icmp-time-exceeded}.) Éste paquete 

\begin{figure}[h]
  \vspace{1em}
  \begin{center}
    \begin{bytefield}[bitwidth=1.1em]{32}
      \bitheader{0-31} \\
      \bitbox{8}{Type = 11} & \bitbox{8}{Code} & \bitbox{16}{Header Checksum} \\
      \bitbox{32}{\emph{No utilizado}} \\
      \bitbox{32}{\emph{Header IP y los primeros 8 bytes de datos del datagrama original}}
    \end{bytefield}
    \caption{Paquete ICMP de tipo Time Exceeded}
  \end{center}
  \label{fig:icmp-time-exceeded}
\end{figure}



\begin{figure}[h]
  \vspace{1em}
  \begin{center}
    \begin{bytefield}[bitwidth=1.1em]{32}
      \bitheader{0-31} \\
      \bitbox{8}{Type = 11} & \bitbox{8}{Code} & \bitbox{16}{Header Checksum} \\
      \bitbox{32}{\emph{No utilizado}} \\
      \begin{rightwordgroup}{Header del \\ paquete original}
        \bitbox{32}{\emph{Header IP}} \\
        \bitbox{8}{Type = 8} & \bitbox{8}{Code = 0} & \bitbox{16}{Header Checksum} \\
        \bitbox{16}{Identifier} & \bitbox{16}{Sequence Number}
      \end{rightwordgroup}
    \end{bytefield}
    \caption{Paquete ICMP de tipo Time Exceeded como respuesta a otro paquete ICMP de tipo Echo Request}
  \end{center}
  \label{fig:icmp-time-exceeded-echo-request}
\end{figure}







%%%%%%%%%%%%%%%%%%%%%%%%%%%%%%%%%%%%%%%%%%%%%%%%%%%%%%%%%%%%%%%%%%%%%%%%%%%%%%%
%% Resultados                                                                %%
%%%%%%%%%%%%%%%%%%%%%%%%%%%%%%%%%%%%%%%%%%%%%%%%%%%%%%%%%%%%%%%%%%%%%%%%%%%%%%%


\section{Resultados}


\subsection{\oxford}

\estadisticas{statistics-www.ox.ac.uk.txt}
             {Traza hacia \oxford}
             {fig:trace-oxford}

\grafico{map-www.ox.ac.uk}
        {Ruta hacia \oxford}
        {fig:map-oxford}

\grafico{rtt-www.ox.ac.uk}
        {RTT de los gateways de la ruta hacia \oxford}
        {fig:rtt-oxford}

\grafico{zrtt-www.ox.ac.uk}
        {ZRTT de los gateways de la ruta hacia \oxford}
        {fig:zrtt-oxford}


\subsection{\sydney}

\estadisticas{statistics-www.sydney.edu.au.txt}
             {Traza hacia \sydney}
             {fig:trace-sydney}

\grafico{map-www.sydney.edu.au}
        {Ruta hacia \sydney}
        {fig:map-sydney}

\grafico{rtt-www.sydney.edu.au}
        {RTT de los gateways de la ruta hacia \sydney}
        {fig:rtt-sydney}

\grafico{zrtt-www.sydney.edu.au}
        {ZRTT de los gateways de la ruta hacia \sydney}
        {fig:zrtt-sydney}


\subsection{\must}

\estadisticas{statistics-www.must.edu.my.txt}
             {Traza hacia \must}
             {fig:trace-must}

\grafico{map-www.must.edu.my}
        {Ruta hacia \must}
        {fig:map-must}

\grafico{rtt-www.must.edu.my}
        {RTT de los gateways de la ruta hacia \must}
        {fig:rtt-must}

\grafico{zrtt-www.must.edu.my}
        {ZRTT de los gateways de la ruta hacia \must}
        {fig:zrtt-must}


%%%%%%%%%%%%%%%%%%%%%%%%%%%%%%%%%%%%%%%%%%%%%%%%%%%%%%%%%%%%%%%%%%%%%%%%%%%%%%%
%% Discusión                      			                                     %%
%%%%%%%%%%%%%%%%%%%%%%%%%%%%%%%%%%%%%%%%%%%%%%%%%%%%%%%%%%%%%%%%%%%%%%%%%%%%%%%


\section{Discusión}

Podemos notar, en los resultados de todas las universidades analizadas, que el segundo nodo, el nodo más próximo a nuestra conexión a internet, posee un RTT promedio alto en comparación a los nodos más próximos a este. Podemos deducir que esto se debe a que este router tiene una prioridad baja para las respuestas a paquetes ICMP, haciendo que las respuestas tarden más de lo esperado.


Otro punto a tener en cuenta, también visible en los resultados de los traceroutes a todas las universidades, es que la biblioteca elegida para la geolocalización basa muy probablemente la localización de los IP según la procedencia de la compañía que realize el enlace entre paises. Puede notarse en los análisis como existen enlaces de bajo ZRTT relativo entre paises y el próximo nodo dentro del país destino tiene un ZRTT relativo demaciado alto, indicando probablemente un enlace de gran distancia.

Un dato estadístico anómalo general es el de obtener un RTT absoluto menor al RTT absoluto del nodo siguiente, lo que significaría que llegar al nodo siguiente tardaría menos que llegar al anterior. Esta anomalía se debe a que los valores del RTT absolutos se calculan mediante el promedio de estos para cada nodo, pudiendo el paquete ICMP haber tomado caminos diferentes y habiendo conseguido llegar de manera a penas más rápida en promedio.



%%%%%%%%%%%%%%%%%%%%%%%%%%%%%%%%%%%%%%%%%%%%%%%%%%%%%%%%%%%%%%%%%%%%%%%%%%%%%%%
%% Conclusión                                                                %%
%%%%%%%%%%%%%%%%%%%%%%%%%%%%%%%%%%%%%%%%%%%%%%%%%%%%%%%%%%%%%%%%%%%%%%%%%%%%%%%


\section{Conclusión}

Pendiente.


%%%%%%%%%%%%%%%%%%%%%%%%%%%%%%%%%%%%%%%%%%%%%%%%%%%%%%%%%%%%%%%%%%%%%%%%%%%%%%%
%% Referencias                                                               %%
%%%%%%%%%%%%%%%%%%%%%%%%%%%%%%%%%%%%%%%%%%%%%%%%%%%%%%%%%%%%%%%%%%%%%%%%%%%%%%%


\begin{thebibliography}{9}

\bibitem{scapy}
	\emph{Scapy Project}.
	\url{http://www.secdev.org/projects/scapy}, 
	Mayo de 2014
	
\end{thebibliography}


\end{document}